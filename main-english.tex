% !TeX spellcheck = en-US
% !TeX encoding = utf8
% !TeX program = pdflatex
% !BIB program = biber
% -*- coding:utf-8 mod:LaTeX -*-


% vv  scroll down to line 200 for content  vv


\let\ifdeutsch\iffalse
\let\ifenglisch\iftrue
\input{pre-documentclass}
\documentclass[
  % fontsize=11pt is the standard
  a4paper,  % Standard format - only KOMAScript uses paper=a4 - https://tex.stackexchange.com/a/61044/9075
  twoside,  % we are optimizing for both screen and two-sided printing. So the page numbers will jump, but the content is configured to stay in the middle (by using the geometry package)
  bibliography=totoc,
  %               idxtotoc,   %Index ins Inhaltsverzeichnis
  %               liststotoc, %List of X ins Inhaltsverzeichnis, mit liststotocnumbered werden die Abbildungsverzeichnisse nummeriert
  headsepline,
  cleardoublepage=empty,
  parskip=half,
  %               draft    % um zu sehen, wo noch nachgebessert werden muss - wichtig, da Bindungskorrektur mit drin
  draft=false
]{scrbook}
\input{config}


\usepackage[
	title={Energy and CO2 footprint of cloud/backend processing},
	author={Erisa Hoxha},
	type=master,
	institute=iaas, % or other institute names - or just a plain string using {Demo\\Demo...}
	course={Computer Science},
	examiner={Prof.\ Dr.\ Marco Aiello},
	supervisor={Prof.\ Dr.\ Marco Aiello},
	startdate={November 22, 2023},
	enddate={May 22, 2024}
]{scientific-thesis-cover}

% Hier stehen alle Abkürzungen
\newacronym{er}{ER}{error rate}
\newacronym{fr}{FR}{Fehlerrate}
\newacronym[plural={RDBMS},shortplural={RDBMS}]{rdbms}{RDBMS}{Relational Database Management System}

\newacronym{ghg}{GHG}{Greenhouse Gases}
\newacronym{it}{IT}{Information Technology}
\newacronym{ai}{AI}{Artificial Intelligence}
\newacronym{aws}{AWS}{Amazon Web Services}
\newacronym{se4s}{SE4S}{Software Engineering for Sustainability}
\newacronym{dc}{DC}{Data Center}
\newacronym{ecer}{ECER}{Energy Consumption and Carbon Emission}
\newacronym{hvac}{HVAC}{Heating, Ventilation, and Air Conditioning}
\newacronym{crac}{CRAC}{Computer Room Air Conditioning}
\newacronym{ashrae}{ASHRAE}{American Society of Heating, Refrigerating and Air-Conditioning Engineers}
\newacronym{ssd}{SSD}{Solid-State Drives}
\newacronym{hdd}{HDD}{Hard Disk Drive}
\newacronym{pue}{PUE}{Power Usage Effectiveness}
\newacronym{vm}{VM}{Virtual Machine}
\newacronym{ict}{ICT}{Information and Communication Technology}
\newacronym{oecd}{OECD}{Organisation for Economic Co-operation and Development}
\newacronym{dcim}{DCIM}{Data Center Infrastructure Management}

\makeindex

\begin{document}

%tex4ht-Konvertierung verschönern
\iftex4ht
  % tell tex4ht to create pictures also for formulas starting with '$'
  % WARNING: a tex4ht run now takes forever!
  \Configure{$}{\PicMath}{\EndPicMath}{}
  %$ % <- syntax highlighting fix for emacs
  \Css{body {text-align:justify;}}

  %conversion of .pdf to .png
  \Configure{graphics*}
  {pdf}
  {\Needs{"convert \csname Gin@base\endcsname.pdf
      \csname Gin@base\endcsname.png"}%
    \Picture[pict]{\csname Gin@base\endcsname.png}%
  }
\fi

%\VerbatimFootnotes %verbatim text in Fußnoten erlauben. Geht normalerweise nicht.

\input{commands}
\pagenumbering{arabic}
\Titelblatt

%Eigener Seitenstil fuer die Kurzfassung und das Inhaltsverzeichnis
\deftriplepagestyle{preamble}{}{}{}{}{}{\pagemark}
%Doku zu deftriplepagestyle: scrguide.pdf
\pagestyle{preamble}
\renewcommand*{\chapterpagestyle}{preamble}



%Kurzfassung / abstract
%auch im Stil vom Inhaltsverzeichnis
\ifdeutsch
  \section*{Kurzfassung}
\else
  \section*{Abstract}
\fi

<Short summary of the thesis>

\cleardoublepage


% BEGIN: Verzeichnisse

\iftex4ht
\else
  \microtypesetup{protrusion=false}
\fi

%%%
% Literaturverzeichnis ins TOC mit aufnehmen, aber nur wenn nichts anderes mehr hilft!
% \addcontentsline{toc}{chapter}{Literaturverzeichnis}
%
% oder zB
%\addcontentsline{toc}{section}{Abkürzungsverzeichnis}
%
%%%

%Produce table of contents
%
%In case you have trouble with headings reaching into the page numbers, enable the following three lines.
%Hint by http://golatex.de/inhaltsverzeichnis-schreibt-ueber-rand-t3106.html
%
%\makeatletter
%\renewcommand{\@pnumwidth}{2em}
%\makeatother
%
\tableofcontents

% Bei einem ungünstigen Seitenumbruch im Inhaltsverzeichnis, kann dieser mit
% \addtocontents{toc}{\protect\newpage}
% an der passenden Stelle im Fließtext erzwungen werden.

\listoffigures
\listoftables

%Wird nur bei Verwendung von der lstlisting-Umgebung mit dem "caption"-Parameter benoetigt
%\lstlistoflistings
%ansonsten:
\ifdeutsch
  \listof{Listing}{Verzeichnis der Listings}
\else
  \listof{Listing}{List of Listings}
\fi

%mittels \newfloat wurde die Algorithmus-Gleitumgebung definiert.
%Mit folgendem Befehl werden alle floats dieses Typs ausgegeben
\ifdeutsch
  \listof{Algorithmus}{Verzeichnis der Algorithmen}
\else
  \listof{Algorithmus}{List of Algorithms}
\fi
%\listofalgorithms %Ist nur für Algorithmen, die mittels \begin{algorithm} umschlossen werden, nötig

% Abkürzungsverzeichnis
\printnoidxglossaries

\iftex4ht
\else
  %Optischen Randausgleich und Grauwertkorrektur wieder aktivieren
  \microtypesetup{protrusion=true}
\fi

% END: Verzeichnisse


% Headline and footline
\renewcommand*{\chapterpagestyle}{scrplain}
\pagestyle{scrheadings}
\pagestyle{scrheadings}
\ihead[]{}
\chead[]{}
\ohead[]{\headmark}
\cfoot[]{}
\ofoot[\usekomafont{pagenumber}\thepage]{\usekomafont{pagenumber}\thepage}
\ifoot[]{}


%% vv  scroll down for content  vv %%































%%%%%%%%%%%%%%%%%%%%%%%%%%%%%%%%%%%%%%%%%%%%%%%%%%%%%%%%%%%%%%%%%%%%%%%%%%%%%%
%
% Main content starts here
%
%%%%%%%%%%%%%%%%%%%%%%%%%%%%%%%%%%%%%%%%%%%%%%%%%%%%%%%%%%%%%%%%%%%%%%%%%%%%%%


\chapter{Introduction}
\section{Motivation}
Climate change is one of the hottest topics of this century. GHG contribute the most in heating of the planet. It is estimated that data centers alone cause 3.7\% of all greenhouse emissions, thus exceeding the emissions from commercial flights in 2022. The demands for IT services and technologies have been increasing with the advancements of machine learning, AI, cloud gaming, streaming etc. Data storage is estimated to reach 175ZB in 2025\cite{zhou2021role}. 
\begin{figure}
	\centering
	\includegraphics[width=\textwidth]{total_data_traffic_zb.png}
	\caption{Total Data Traffic Forecast through 2030. Source: “Impact of AI on Electronics and Semiconductor Industries”, IBS, April 2020}
	\label{fig:chor1}
\end{figure}
This growth of data processing and data storage has improved the quality of our life, but at the same time hundreds of MtCO2 are released yearly from this. As of today, USA has the largest number of data centers in the world with more than 2000 sites, thus causing highest carbon footprint per country based. While in the beginning of the century these data centers were not build thinking about their environmental impact, this has seemed to change in the recent years. There have been improvements in the way data centers are build, where they are build and there has been refinement in the hardware used. So the same algorithm can be executed in less power consumption. 
\begin{figure}
	\centering
	\includegraphics[width=\textwidth]{datacentermap.jpg}
	\caption{Data Center distribution worldwide}
	\label{fig:chor1}
\end{figure}
Many institutions have already migrated their services to clouds and decreasing their overall footprint. With big cloud providers like AWS, companies can decide where they want to host their services and can track their environmental effect while the resources are being managed more efficiently.Big companies have advanced in reducing carbon release and Google today uses 100\% green energy. But other smaller companies still do not have the right mastery energy management. 

Instructions that that a computer what actions to perform are mainly softwares\cite{Britannica_2024}. Software does not consume energy by itself directly, it directs and influences the operations of hardware, therefore causing carbon emissions. In a survey committed by programmers, only 18\% of them stated that they build these programs with efficiency in mind. Considerable work has been done in the development of software with sustainability at its core, thanks to the advancements made in Software Engineering for Sustainability (SE4S). Nevertheless, since this is a relatively new field of study, there is still a significant amount of work to be carried out in this area.


%It will therefore be important to address sustainability holistically across the various components of total lifecycle exergy.
%softwares are not optimized
IT in some way has improved our lives in terms of: Life Satisfaction (Trust and Safety), Mobility (Culture and Sports), Integration (Sustainability) and Public Services\cite{nevado2019improving}. Now we can attend a company meeting from the comfort of our own homes instead of commuting to office and therefore lowering overall carbon release. Also there exist data centers that are just tackling the CO2 release itself. So, IT has done more good in terms of carbon emissions, but with the newest researches, the situation can be improved a lot still in terms of electricity production, space and thermal management. GeSISmarter report from 2020 stated the that greenhouse gas emissions can be reduced by 16.5\% (9.1GtCO2) with effective use of IT\cite{cosar2019carbon}. 


\section{Problem statement}

In 2006, the British mathematician and entrepreneur Clive Humby coined the phrase “Data is the new oil” \cite{humby2006data}. Michael Palmer expanded on Humby's quote by saying, "like oil, data is valuable, but if unrefined it cannot really be used [Oil] has to be changed into gas, plastic, chemicals, etc to create a valuable entity that drives profitable activity; so, data must be broken down and analyzed for it to have value”\cite{palmer2006}. As of March 2024 there are currently 10593 data centers worldwide with US leading this with 5381 data centers\cite{statista_2024}. 

\begin{figure}
	\centering
	\includegraphics[width=\textwidth]{leading-countries-by-number-of-data-centers-2024.png}
	\caption{Source Statista\cite{statista_2024}}
	\label{fig:chor1}
\end{figure}

Data centers constructed in the past did not take into consideration the environmental considerations. However, as the demand for various products, systems, and activities grew, it became evident that they had a significant carbon footprint. \cite{zhu2023future} concluded:
\begin{enumerate}
	\item Approximately 20–40\% of the energy consumed by DC can be reduced by optimizing IT equipment, including server layout adjustment, virtualization technology, and storage equipment improvement.
	\item Approximately 15–27\% of the energy consumed by DC can be reduced via research and development of advanced cooling technologies, including natural cooling and liquid cooling to achieve sustainability.
	\item DCs' ECER strategies are affected by the differences in geographical location, natural resources and economic basis, and the single energy-saving method is difficult to meet the goal of zero-carbon emission.
	\item The key challenges and potential opportunities in the future decarbonization path of DCs are summarized from the perspectives of policy reform, technological innovation, and resource diversification and management, which is great significance to the realization of zero-carbon DCs and the sustainable development of human society.
\end{enumerate}


% too many existing data centers
% many new data centers are being build
% large contribution in overall worldwide footprint
% resources should be better managed
% high demand of machine learning, ai, machine learning, video streaming and blockchain
% to produce energy we can use renewable energy resources
% cooling can also be guaranteed from different meas

\section{Structure}

In the following, the structure is outlines as follows:


\begin{itemize}
	\item Chapter 2 is the background of this all this and has three subpoints: Energy, Greenhouse Gases and Data Centers
	\item Chapter 3 is the Study Design along where I included my research questions and the methodology in my research used to find answers for those questions
	\item Chapter 4 is about all about Energy. In this chapter I studied separately the energy used for running servers and energy used for cooling
	\item Chapter 5 is the hardware footprint of data centers in years
	\item Chapter 6 is about Cloud and Virtualization and answers the research question 4
	\item Chapter 7 is about Energy saved from data centers and clouds and the concept of Carbon-Free Energy Data centers
	\item Chapter 8 - here there are the conclusions from my my research 
\end{itemize}

%This thesis starts with \cref{chap:k2}.

%We can also typeset \verb|<text>verbatim text</text>|.
%Backticks are also rendered correctly: \verb|`words in backticks`|.

\chapter{Background}

We start with providing insights into relevant background knowledge of this thesis. The relevant topics include: "data center", "data center carbon footprint", "data center energy consumption", "data center sustainability", "cloud efficiency", " energy conservation" and " emission reduction technology", " server thermal management", "cooling in data centers", "liquid cooling". 


\section{Energy}
The energy consumption of servers serves as the fundamental metric for assessing power and heat flow within data centers. In these environments, the interplay between IT equipment and cooling systems is tightly intertwined due to thermal considerations. In the context of China, data centers have reached a notable level of energy consumption, with their combined usage ranging between 120 and 130 billion kWh. This accounts for approximately 2\% of the total electricity consumption in the country. Similarly, in 2014, data centers in the United States consumed an estimated 70 billion kWh, representing roughly 1.8\% of the nation's overall electricity usage.\cite{JIN2020114806}. 


The data center's energy consumption primarily stems from two main components: IT equipment and cooling equipment, which collectively account for approximately 90\% of the total energy usage\cite{JIN2020114806}. Figure 2.2 provides an illustration of the dynamics of energy and air flows within a data center. This visual representation offers valuable insights into the aspects of data center operations and highlights the interplay between energy consumption and airflow management. IT equipment, including servers, storage devices, and networking infrastructure, represents a significant portion of the energy consumed within a data center. These electronic devices require power to operate efficiently and process the vast amounts of data they handle. The energy consumed by IT equipment is influenced by factors such as the number of devices, their processing power, and their utilization rates\cite{von2016variations}. Additionally, cooling equipment plays a crucial role in maintaining optimal operating conditions within the data center environment. As IT equipment generates heat during operation, cooling systems, such as air conditioning units and precision cooling systems, are employed to regulate temperatures and prevent overheating. The energy consumed by cooling equipment is necessary to ensure the reliability and performance of the IT infrastructure\cite{rong2016optimizing}.

\begin{figure}
	\centering
	\includegraphics[width=\textwidth]{pie_chart_energy_draw.png}
	\caption{Analysis of a typical 465 m2 data center, Source Emerson Network power\cite{emerson2015}}
\end{figure}



\begin{figure}
	\centering
	\includegraphics[width=\textwidth]{energy_management.jpg}
	\caption{The power flow and heat flow in general data centers \cite{JIN2020114806}}
\end{figure}
%\includegraphics{graphics/energy_management.jpg}

Energy can be generated from various sources. Some of the most common methods of energy generation:

Fossil Fuels: Fossil fuels such as coal, oil, and natural gas are burned to produce heat, which is then used to generate electricity. This process is commonly used in power plants\cite{owid-fossil-fuels}.

Nuclear Energy: Nuclear power plants use a process called nuclear fission to generate heat. The heat produced by splitting atoms is used to create steam, which drives turbines and generates electricity\cite{ritchie2023nuclear}.

Renewable Sources: Renewable energy sources include solar, wind, hydroelectric, geothermal, and biomass. Solar energy is generated by converting sunlight into electricity using photovoltaic cells. Wind energy is harnessed by wind turbines that convert the kinetic energy of the wind into electricity. Hydroelectric power is generated by capturing the energy of flowing or falling water. Geothermal energy utilizes the heat from the Earth's interior to generate electricity. Biomass energy is produced by burning organic matter such as wood, agricultural crops, or waste materials\cite{owid-renewable-energy}.

Hydropower: Hydropower is a specific type of renewable energy that generates electricity through the force or energy of moving water, such as rivers or waterfalls. It involves the use of dams or flow-through turbines to convert the kinetic energy of moving water into electrical energy\cite{sipahutar2013renewable}.

Tidal and Wave Energy: Tidal energy is generated by harnessing the power of ocean tides, while wave energy is generated by capturing the energy of ocean waves. Both methods involve specialized technologies to convert the mechanical energy of water into electricity\cite{khan2017review}.

Fuel Cells: Fuel cells generate electricity through an electrochemical process, usually by combining hydrogen with oxygen to produce water and electricity. Fuel cells can use hydrogen derived from various sources, including natural gas, biomass, or renewable energy\cite{mekhilef2012comparative}.

It's important to note that the availability and utilization of different energy sources can vary depending on factors such as geographical location, technological advancements, and economic considerations\cite{firozjaei2020effect}\cite{kryzia2019dampening}\cite{abas2015review}. The transition toward renewable and sustainable energy sources is gaining momentum globally due to concerns about climate change and the finite nature of fossil fuel resources\cite{qazi2019towards}.

\section{Greenhouse gases}
United Nations Climate Change Conference (COP26) held in Glasgow reached an agreement to ensure global net zero emissions by mid-century and to reduce global emissions by 45\% by 2030\cite{Arora_Mishra_2021}. 


Greenhouse gases are gases in the atmosphere that absorb and emit radiation within the thermal infrared range. This process is the fundamental cause of the greenhouse effect. Carbon dioxide (chemical formula CO2) is an important greenhouse gas, which contributes 9\%–26\% of the greenhouse effect. 

The greenhouse gasses included in CO2e calculations are\cite{dioxide2017overview}:
\begin{itemize}
	\item Carbon dioxide (CO2)
	\item Nitrous oxide (N2O)
	\item Methane (CH4)
	\item Fluorinated Gases
\end{itemize}

A data center consumes significant amount of power and a mass of greenhouse gas is produced in the process of power generation. According to US Energy Information Administration\cite{US2023}, about 0.86 pounds (0.39 kg) of CO2 is released per kWh. Below is the descriptive map greenhouse emissions for each country. 

\begin{figure}
	\centering
	\includegraphics[width=\textwidth]{total-ghg-emissions.png}
	\caption{Greenhouse gas emissions, 2022 \cite{owid-co2-and-greenhouse-gas-emissions}}
\end{figure}


Measuring CO2 emissions from data center computing presents challenges due to the intricate nature of data center infrastructure and the various factors that influence CO2 production, including data center efficiency and energy sources used\cite{wang2013review}. \cite{review_data_center_2021} estimated that 720 million tons of CO2 emissions will be released by data centers only in 2030. The amount of CO2 generated by data centers is influenced by multiple factors, with data center efficiency and energy sources being key variables. These variables can vary significantly across different data centers, making it difficult to accurately measure CO2 emissions. Additionally, data centers are complex environments with shared infrastructure utilized by multiple users and managed support systems, further complicating the precise calculation of CO2 emissions attributed to individual applications, users, or computing servers.

\section{Data centers}

%\includegraphics{graphics/datacentermap.jpg}
A data center is a physical room, building or facility that houses IT infrastructure for building, running, and delivering applications and services, and for storing and managing the data associated with those applications and services\cite{ibmWhatData}.


In Figure 2.4, we can observe the hierarchical structure of the data center (DC), which is divided into three distinct layers. Starting from the top layer, we find the core switches responsible for receiving data service instructions and transmitting them to the front-end servers through the network. Moving down to the middle layer, we encounter the aggregation switches, which serve as the connection point between the top and bottom layers. Their primary function is to facilitate the consolidation of data from various sources. Finally, at the base layer, we have the front-end and back-end servers. The front-end servers handle user instructions and requests, while the back-end servers allocate storage nodes\cite{ZHU2023104322}.

The main objective of the DC is to effectively integrate and centralize network and storage resources using virtualization technology. This integration allows for efficient data processing. Additionally, the DC employs network node virtual functions to monitor, manage, and oversee the performance of individual nodes within the network. By leveraging these technologies, the DC is able to streamline operations and optimize resource allocation for enhanced performance\cite{ZHU2023104322}.


\begin{figure}
	\centering
	\includegraphics[width=\textwidth]{data_center_structure.jpg}
	\caption{Diagrammatic representation of the DC's overall structure\cite{ZHU2023104322}}
\end{figure}

Spending on data center systems is expected to see a notable jump in growth from 2023 (4\%) to 2024 (10\%), in large part due to planning for generative AI GenAI\cite{Gartner2024}.


\blinddocument

\chapter{Related Work}

Describe relevant scientific literature related to your work.

\chapter{Conclusion and Outlook}
\label{chap:zusfas}

\section*{Outlook}

\printbibliography

All links were last followed on March 17, 2018.

\appendix
\input{latexhints-english}

\pagestyle{empty}
\renewcommand*{\chapterpagestyle}{empty}
\Versicherung
\end{document}
