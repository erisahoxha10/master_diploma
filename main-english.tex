% !TeX spellcheck = en-US
% !TeX encoding = utf8
% !TeX program = pdflatex
% !BIB program = biber
% -*- coding:utf-8 mod:LaTeX -*-


% vv  scroll down to line 200 for content  vv


\let\ifdeutsch\iffalse
\let\ifenglisch\iftrue
\input{pre-documentclass}
\documentclass[
  % fontsize=11pt is the standard
  a4paper,  % Standard format - only KOMAScript uses paper=a4 - https://tex.stackexchange.com/a/61044/9075
  twoside,  % we are optimizing for both screen and two-sided printing. So the page numbers will jump, but the content is configured to stay in the middle (by using the geometry package)
  bibliography=totoc,
  %               idxtotoc,   %Index ins Inhaltsverzeichnis
  %               liststotoc, %List of X ins Inhaltsverzeichnis, mit liststotocnumbered werden die Abbildungsverzeichnisse nummeriert
  headsepline,
  cleardoublepage=empty,
  parskip=half,
  %               draft    % um zu sehen, wo noch nachgebessert werden muss - wichtig, da Bindungskorrektur mit drin
  draft=false
]{scrbook}
\input{config}


\usepackage[
	title={Energy and CO2 footprint of cloud/backend processing},
	author={Erisa Hoxha},
	type=master,
	institute=iaas, % or other institute names - or just a plain string using {Demo\\Demo...}
	course={Computer Science},
	examiner={Prof.\ Dr.\ Marco Aiello},
	supervisor={Prof.\ Dr.\ Marco Aiello},
	startdate={November 22, 2023},
	enddate={May 21, 2024}
]{scientific-thesis-cover}

% Hier stehen alle Abkürzungen
\newacronym{er}{ER}{error rate}
\newacronym{fr}{FR}{Fehlerrate}
\newacronym[plural={RDBMS},shortplural={RDBMS}]{rdbms}{RDBMS}{Relational Database Management System}

\newacronym{ghg}{GHG}{Greenhouse Gases}
\newacronym{it}{IT}{Information Technology}
\newacronym{ai}{AI}{Artificial Intelligence}
\newacronym{aws}{AWS}{Amazon Web Services}
\newacronym{se4s}{SE4S}{Software Engineering for Sustainability}
\newacronym{dc}{DC}{Data Center}
\newacronym{ecer}{ECER}{Energy Consumption and Carbon Emission}
\newacronym{hvac}{HVAC}{Heating, Ventilation, and Air Conditioning}
\newacronym{crac}{CRAC}{Computer Room Air Conditioning}
\newacronym{ashrae}{ASHRAE}{American Society of Heating, Refrigerating and Air-Conditioning Engineers}
\newacronym{ssd}{SSD}{Solid-State Drives}
\newacronym{hdd}{HDD}{Hard Disk Drive}
\newacronym{pue}{PUE}{Power Usage Effectiveness}
\newacronym{vm}{VM}{Virtual Machine}
\newacronym{ict}{ICT}{Information and Communication Technology}
\newacronym{oecd}{OECD}{Organisation for Economic Co-operation and Development}
\newacronym{dcim}{DCIM}{Data Center Infrastructure Management}

\makeindex

\begin{document}

%tex4ht-Konvertierung verschönern
\iftex4ht
  % tell tex4ht to create pictures also for formulas starting with '$'
  % WARNING: a tex4ht run now takes forever!
  \Configure{$}{\PicMath}{\EndPicMath}{}
  %$ % <- syntax highlighting fix for emacs
  \Css{body {text-align:justify;}}

  %conversion of .pdf to .png
  \Configure{graphics*}
  {pdf}
  {\Needs{"convert \csname Gin@base\endcsname.pdf
      \csname Gin@base\endcsname.png"}%
    \Picture[pict]{\csname Gin@base\endcsname.png}%
  }
\fi

%\VerbatimFootnotes %verbatim text in Fußnoten erlauben. Geht normalerweise nicht.

\input{commands}
\pagenumbering{arabic}
\Titelblatt

%Eigener Seitenstil fuer die Kurzfassung und das Inhaltsverzeichnis
\deftriplepagestyle{preamble}{}{}{}{}{}{\pagemark}
%Doku zu deftriplepagestyle: scrguide.pdf
\pagestyle{preamble}
\renewcommand*{\chapterpagestyle}{preamble}



%Kurzfassung / abstract
%auch im Stil vom Inhaltsverzeichnis
\ifdeutsch
  \section*{Kurzfassung}
\else
  \section*{Abstract}
\fi

	Data centers are the backbone of the digital infrastructure and they are responsible for storing and processing large amounts of data. In 2023, global emissions from cloud computing made up from 2.5\% to 3.7\% of worldwide GHG, thereby exceeding emissions from commercial flights (2.4\%). The demand for data centers is increasing exponentially due to high number of users (5.8 billion in October 2023) and rising demand for services and technologies such as streaming, cloud gaming, blockchain, artificial intelligence and machine learning. The emission in this data centers is caused by electricity consumption (to run the servers), water consumption (to run the servers) and the lifetime of the equipment. Many companies have outsourced their tech infrastructure thus their information is being managed more efficiently, but this measure does not remove completely their environmental impact. Different measures are currently being taken from the top three cloud providers: AWS, Microsoft Azure and GCP. AWS leads the list with over 30\% market share and they have committed to reaching net zero emissions by 2040. They announced that they will inform their customers about their carbon footprint while continuing to improve emissions release. Azure team is also working to improve their climate impact and are currently trying on a 2-year-old experiment where they have submerged 800 servers on the ocean floor of the Scottish coast. On the other side, GPC, which has the least market share among these three, claims to have already reached net zero emissions. Apart from there, there are many software that are using different algorithm to measure the carbon footprint for data centers.


\cleardoublepage


% BEGIN: Verzeichnisse

\iftex4ht
\else
  \microtypesetup{protrusion=false}
\fi

%%%
% Literaturverzeichnis ins TOC mit aufnehmen, aber nur wenn nichts anderes mehr hilft!
% \addcontentsline{toc}{chapter}{Literaturverzeichnis}
%
% oder zB
%\addcontentsline{toc}{section}{Abkürzungsverzeichnis}
%
%%%

%Produce table of contents
%
%In case you have trouble with headings reaching into the page numbers, enable the following three lines.
%Hint by http://golatex.de/inhaltsverzeichnis-schreibt-ueber-rand-t3106.html
%
%\makeatletter
%\renewcommand{\@pnumwidth}{2em}
%\makeatother
%
\tableofcontents

% Bei einem ungünstigen Seitenumbruch im Inhaltsverzeichnis, kann dieser mit
% \addtocontents{toc}{\protect\newpage}
% an der passenden Stelle im Fließtext erzwungen werden.

%\listoffigures
%\listoftables

%Wird nur bei Verwendung von der lstlisting-Umgebung mit dem "caption"-Parameter benoetigt
%\lstlistoflistings
%ansonsten:
\ifdeutsch
  \listof{Listing}{Verzeichnis der Listings}
\else
  %\listof{Listing}{List of Listings}
\fi

%mittels \newfloat wurde die Algorithmus-Gleitumgebung definiert.
%Mit folgendem Befehl werden alle floats dieses Typs ausgegeben
\ifdeutsch
  \listof{Algorithmus}{Verzeichnis der Algorithmen}
\else
  %\listof{Algorithmus}{List of Algorithms}
\fi
%\listofalgorithms %Ist nur für Algorithmen, die mittels \begin{algorithm} umschlossen werden, nötig

% Abkürzungsverzeichnis
\printnoidxglossaries

\iftex4ht
\else
  %Optischen Randausgleich und Grauwertkorrektur wieder aktivieren
  \microtypesetup{protrusion=true}
\fi

% END: Verzeichnisse


% Headline and footline
\renewcommand*{\chapterpagestyle}{scrplain}
\pagestyle{scrheadings}
\pagestyle{scrheadings}
\ihead[]{}
\chead[]{}
\ohead[]{\headmark}
\cfoot[]{}
\ofoot[\usekomafont{pagenumber}\thepage]{\usekomafont{pagenumber}\thepage}
\ifoot[]{}


%% vv  scroll down for content  vv %%































%%%%%%%%%%%%%%%%%%%%%%%%%%%%%%%%%%%%%%%%%%%%%%%%%%%%%%%%%%%%%%%%%%%%%%%%%%%%%%
%
% Main content starts here
%
%%%%%%%%%%%%%%%%%%%%%%%%%%%%%%%%%%%%%%%%%%%%%%%%%%%%%%%%%%%%%%%%%%%%%%%%%%%%%%


\chapter{Introduction}
\section{Motivation}
The issue of climate change has been a topic of discussion in the present day. \gls{ghg} contribute the most in heating of the planet. It is estimated that data centers alone cause 3.7\% of all greenhouse emissions, thus exceeding the emissions from commercial flights in 2022. The demand for \gls{it} services and technologies has been increasing with the advancements of machine learning, \gls{ai}, cloud gaming, streaming etc. Data storage is estimated to reach 175 ZB in 2025\cite{zhou2021role} as figure \ref{total_data_traffic} shows. 
\begin{figure}
	\centering
	\includegraphics[width=\textwidth]{total_data_traffic_zb.png}
	\caption{Total Data Traffic Forecast through 2030. Source: “Impact of AI on Electronics and Semiconductor Industries”, IBS, April 2020}
	\label{total_data_traffic}
\end{figure}
This growth of data processing and data storage has improved the quality of our life, but at the same time hundreds of MtCO2 are released yearly from this. As of today, USA has the largest number of data centers in the world with more than 2000 sites, thus causing highest carbon footprint per country based. Figure \ref{datacentermap} shows a distribution of data centers worldwide. While in the beginning of the century these data centers were not build thinking about their environmental impact, this has seemed to change in the recent years. There have been improvements in the way data centers are build, where they are build and there has been refinement in the hardware used. So the same algorithm can be executed in less power consumption. 
\begin{figure}
	\centering
	\includegraphics[width=\textwidth]{datacentermap.jpg}
	\caption{Data Center distribution worldwide; Source: Data Center Map \cite{datacentermapDataCenters}}
	\label{datacentermap}
\end{figure}
Many institutions have already migrated their services to clouds and decreasing their overall footprint. With big cloud providers like \gls{aws}, companies can decide where they want to host their services and can track their environmental effect while the resources are being managed more efficiently. Big companies have advanced in reducing carbon release and Google today uses 100\% green energy\cite{blogAccelerateAdvanced}. But other smaller companies still do not have the right mastery energy in  management. 

Software that runs on servers are indirectly the cause of energy consumption, as it tells the computer what actions to perform\cite{Britannica_2024}. Software does not consume energy by itself directly, but it directs and influences the operations of hardware, therefore causing carbon emissions. In a survey committed by programmers, only 18\% of them stated that they build these programs with efficiency in mind\cite{pang2015programmers}. Considerable work has been done in the development of software with sustainability at its core, thanks to the advancements made in \gls{se4s} \cite{swacha2022models}. Nevertheless, since this is a relatively new field of study, there is still a significant amount of work to be carried out in this area.


%It will therefore be important to address sustainability holistically across the various components of total lifecycle exergy.
%softwares are not optimized
IT in some way has improved our lives in terms of: Life Satisfaction (Trust and Safety), Mobility (Culture and Sports), Integration (Sustainability) and Public Services\cite{nevado2019improving}. Now we can attend a company meeting from the comfort of our own homes instead of commuting to office and therefore lowering overall carbon release. Also there exist data centers that are just tackling the CO2 release itself. So, IT has done more good in terms of carbon emissions, but with the newest researches, the situation can be improved a lot still in terms of electricity production, space and thermal management. GeSISmarter report from 2020 stated the that greenhouse gas emissions can be reduced by 16.5\% (9.1GtCO2) with effective use of \gls{it} \cite{cosar2019carbon}. 


\section{Problem Statement}

In 2006, the British mathematician and entrepreneur Clive Humby coined the phrase “Data is the new oil” \cite{humby2006data}. Michael Palmer expanded on Humby's quote by saying, "like oil, data is valuable, but if unrefined it cannot really be used [Oil] has to be changed into gas, plastic, chemicals, etc to create a valuable entity that drives profitable activity; so, data must be broken down and analyzed for it to have value”\cite{palmer2006}. As of March 2024 there are currently 10593 data centers worldwide with US leading this with 5381 data centers\cite{statista_2024}. 

\begin{figure}
	\centering
	\includegraphics[width=\textwidth]{leading-countries-by-number-of-data-centers-2024.png}
	\caption{Leading countries by the number of data centers; Source Statista\cite{statista_2024}}
\end{figure}

Data centers constructed in the past did not take into consideration the environmental considerations. However, as the demand for various products, systems, and activities grew, it became evident that they had a significant carbon footprint. \cite{zhu2023future} concluded:
\begin{enumerate}
	\item Approximately 20–40\% of the energy consumed by \gls{dc} can be reduced by optimizing \gls{it} equipment, including server layout adjustment, virtualization technology, and storage equipment improvement.
	\item Approximately 15–27\% of the energy consumed by DC can be reduced via research and development of advanced cooling technologies, including natural cooling and liquid cooling to achieve sustainability.
	\item \gls{dc}s' \gls{ecer} strategies are affected by the differences in geographical location, natural resources and economic basis, and the single energy-saving method is difficult to meet the goal of zero-carbon emission.
	\item The key challenges and potential opportunities in the future decarbonization path of \gls{dc}s are summarized from the perspectives of policy reform, technological innovation, and resource diversification and management, which is great significance to the realization of zero-carbon \gls{dc}s and the sustainable development of human society.
\end{enumerate}


% too many existing data centers
% many new data centers are being build
% large contribution in overall worldwide footprint
% resources should be better managed
% high demand of machine learning, ai, machine learning, video streaming and blockchain
% to produce energy we can use renewable energy resources
% cooling can also be guaranteed from different meas

\section{Structure}

In the following, the structure is outlined as follows:

%change it to third person - recommended by Ruben


\begin{itemize}
	\item Chapter 2 is the Background. It contains 3 subsections: \enquote{Energy}, \enquote{Greenhouse Gases} and \enquote{Data Centers}.
	\item Chapter 3 is the Study Design where I included my research questions and the methodology I used in this research to answer the 5 research questions
	\item Chapter 4 is Energy used in Data Centers. In this chapter I studied separately the energy used for running servers and energy used for thermal management
	\item Chapter 5 is the Hardware Footprint and E-waste of data centers. In this chapter it is included also the study of hardware recycling
	\item Chapter 6 is about Clouds and Virtualization. In this chapter it is included PUE of different cloud providers
	\item Chapter 7 is about Energy saved from data centers and clouds and the concept of Carbon-Free Energy Data centers. Here are included real life scenarios of how clouds have changed industries
	\item Chapter 8 is about \gls{dcim} software and how it helps with carbon footprint
	\item Chapter 9 research questions are addressed 
\end{itemize}

%This thesis starts with \cref{chap:k2}.

%We can also typeset \verb|<text>verbatim text</text>|.
%Backticks are also rendered correctly: \verb|`words in backticks`|.

\chapter{Background}

This start with providing insights into relevant background knowledge for this thesis. The relevant topics include: "data center", "data center carbon footprint", "data center energy consumption", "data center sustainability", "cloud efficiency", \enquote{energy conservation}, \enquote{emission reduction technology}, \enquote{server thermal management}, "cooling in data centers", "liquid cooling". 


\section{Energy}
The energy consumption of servers serves as the fundamental metric for assessing power and heat flow within data centers. In these environments, the interplay between \gls{it} equipment and cooling systems is tightly intertwined due to thermal considerations. In the context of China, data centers have reached a notable level of energy consumption, with their combined usage ranging between 120 and 130 billion kWh. This accounts for approximately 2\% of the total electricity consumption in the country. Similarly, in 2014, data centers in the United States consumed an estimated 70 billion kWh, representing roughly 1.8\% of the nation's overall electricity usage\cite{JIN2020114806}. 


The data center's energy consumption primarily stems from two main components: \gls{it} equipment and cooling equipment, which collectively account for approximately 90\% of the total energy usage\cite{JIN2020114806}. Figure \ref{power_draw} shows the distribution of power consumed by a \gls{dc}.

\begin{figure}
	\centering
	\includegraphics[width=\textwidth]{pie_chart_energy_draw.png}
	\caption{Analysis of a typical 465 m2 data center, Source Emerson Network power\cite{emerson2015}}
	\label{power_draw}
\end{figure}


 Figure 2.2 provides an illustration of the dynamics of energy and air flows within a data center. This visual representation offers valuable insights into the aspects of data center operations and highlights the interplay between energy consumption and airflow management. \gls{it} equipment, including servers, storage devices, and networking infrastructure, represents a significant portion of the energy consumed within a data center. These electronic devices require power to operate efficiently and process the vast amounts of data they handle. The energy consumed by \gls{it} equipment is influenced by factors such as the number of devices, their processing power, and their utilization rates\cite{von2016variations}. Additionally, cooling equipment plays a crucial role in maintaining optimal operating conditions within the data center environment. As \gls{it} equipment generates heat during operation, cooling systems, such as air conditioning units and precision cooling systems, are employed to regulate temperatures and prevent overheating. The energy consumed by cooling equipment is necessary to ensure the reliability and performance of the \gls{it} infrastructure\cite{rong2016optimizing}.



\begin{figure}
	\centering
	\includegraphics[width=\textwidth]{energy_management.jpg}
	\caption{The power flow and heat flow in general data centers \cite{JIN2020114806}}
\end{figure}
%\includegraphics{graphics/energy_management.jpg}

Energy can be generated from various sources. Some of the most common methods of energy generation are:

\begin{description}
	\item[Fossil Fuels] Fossil fuels such as coal, oil, and natural gas are burned to produce heat, which is then used to generate electricity. This process is commonly used in power plants\cite{owid-fossil-fuels}.
	
	\item[Nuclear Energy] Nuclear power plants use a process called nuclear fission to generate heat. The heat produced by splitting atoms is used to create steam, which drives turbines and generates electricity\cite{ritchie2023nuclear}.
	
	\item[Renewable Sources] Renewable energy sources include solar, wind, hydroelectric, geothermal, and biomass. Solar energy is generated by converting sunlight into electricity using photovoltaic cells. Wind energy is harnessed by wind turbines that convert the kinetic energy of the wind into electricity. Hydroelectric power is generated by capturing the energy of flowing or falling water. Geothermal energy utilizes the heat from the Earth's interior to generate electricity. Biomass energy is produced by burning organic matter such as wood, agricultural crops, or waste materials\cite{owid-renewable-energy}.
	
	\item[Hydropower] Hydropower is a specific type of renewable energy that generates electricity through the force or energy of moving water, such as rivers or waterfalls. It involves the use of dams or flow-through turbines to convert the kinetic energy of moving water into electrical energy\cite{sipahutar2013renewable}.
	
	\item[Tidal and Wave Energy] Tidal energy is generated by harnessing the power of ocean tides, while wave energy is generated by capturing the energy of ocean waves. Both methods involve specialized technologies to convert the mechanical energy of water into electricity\cite{khan2017review}.
	
	\item[Fuel Cells] Fuel cells generate electricity through an electrochemical process, usually by combining hydrogen with oxygen to produce water and electricity. Fuel cells can use hydrogen derived from various sources, including natural gas, biomass, or renewable energy\cite{mekhilef2012comparative}.
\end{description}


It's important to note that the availability and utilization of different energy sources can vary depending on factors such as geographical location, technological advancements, and economic considerations\cite{firozjaei2020effect}\cite{kryzia2019dampening}\cite{abas2015review}. The transition toward renewable and sustainable energy sources is gaining momentum globally due to concerns about climate change and the finite nature of fossil fuel resources\cite{qazi2019towards}.

\section{Greenhouse Gases}\label{ghg_section_background}
United Nations Climate Change Conference (COP26) held in Glasgow reached an agreement to ensure global net zero emissions by mid-century and to reduce global emissions by 45\% by 2030\cite{Arora_Mishra_2021}. 


Greenhouse gases are gases in the atmosphere that absorb and emit radiation within the thermal infrared range. This process is the fundamental cause of the greenhouse effect. Carbon dioxide (chemical formula CO2) is an important greenhouse gas, which contributes 9\%–26\% of the greenhouse effect. 

The greenhouse gasses included in CO2 calculations are\cite{dioxide2017overview}:
\begin{itemize}
	\item Carbon dioxide (CO2)
	\item Nitrous oxide (N2O)
	\item Methane (CH4)
	\item Fluorinated Gases
\end{itemize}

A data center consumes significant amount of power, therefore a mass of greenhouse gas is produced in this process. According to US Energy Information Administration\cite{US2023}, about 0.86 pounds (0.39 kg) of CO2 is released per kWh. \ref{ghg_emissions} is the descriptive map greenhouse emissions for each country. 

\begin{figure}
	\centering
	\includegraphics[width=\textwidth]{total-ghg-emissions.png}
	\caption{Greenhouse gas emissions, 2022 \cite{owid-co2-and-greenhouse-gas-emissions}}
	\label{ghg_emissions}
\end{figure}


Measuring CO2 emissions from data centers presents challenges due to the intricate nature of data center infrastructure and the various factors that influence CO2 production, including data center efficiency and energy sources used\cite{wang2013review}. \cite{review_data_center_2021} estimated that 720 million tons of CO2 emissions will be released by data centers only in 2030. The amount of CO2 generated by data centers is influenced by multiple factors, with data center efficiency and energy sources being key variables. These variables can vary significantly across different data centers, making it difficult to accurately measure CO2 emissions. Additionally, data centers are complex environments with shared infrastructure utilized by multiple users and managed support systems, further complicating the precise calculation of CO2 emissions attributed to individual applications, users, or computing servers.

\section{Data Centers}

%\includegraphics{graphics/datacentermap.jpg}
A data center is a physical room, building or facility that houses \gls{it} infrastructure for building, running, and delivering applications and services, and for storing and managing the data associated with those applications and services\cite{ibmWhatData}.


In Figure \ref{dc_structure}, we can observe the hierarchical structure of the \gls{dc}, which is divided into three distinct layers. Starting from the top layer, we find the core switches responsible for receiving data service instructions and transmitting them to the front-end servers through the network. Moving down to the middle layer, we encounter the aggregation switches, which serve as the connection point between the top and bottom layers. Their primary function is to facilitate the consolidation of data from various sources. Finally, at the base layer, we have the front-end and back-end servers. The front-end servers handle user instructions and requests, while the back-end servers allocate storage nodes\cite{ZHU2023104322}. 


The main objective of the \gls{dc} is to effectively integrate and centralize network and storage resources using virtualization technology. This integration allows for efficient data processing. Additionally, the \gls{dc} employs network node virtual functions to monitor, manage, and oversee the performance of individual nodes within the network. By leveraging these technologies, the DC is able to streamline operations and optimize resource allocation for enhanced performance\cite{ZHU2023104322}.


\begin{figure}
	\centering
	\includegraphics[width=\textwidth]{data_center_structure.jpg}
	\caption{Diagrammatic representation of the DC's overall structure\cite{ZHU2023104322}}
	\label{dc_structure}
\end{figure}

Spending on data center systems is expected to see a notable jump in growth from 2023 (4\%) to 2024 (10\%), in large part due to planning for generative \gls{ai} GenAI\cite{Gartner2024}.


%%%%%%    chapter 3 %%%%%%%%%%%%%%%
\chapter{Study Design}
\section{Research Questions}

\begin{description}
	\item[RQ1] How much \gls{ghg} do data centers currently produce, and what can be an estimate of the \gls{ghg} production for the upcoming years?
	\item[RQ2] Study of the emission causes in data centers, focusing on factors like electricity, cooling and physical hardware fragments.
	\item[RQ3] Introduction to carbon footprint and study of the (existing) measures for the reduction of carbon footprint. 
	\item[RQ4] How can outsourcing to cloud computing reduce the environmental impact?
	\item[RQ5] Study of the DCIM Software for carbon footprint measuring and evaluation of the software's accuracy via experiments.
\end{description}

\section{Methodology}

This section describes the methods used to answer the research questions. 
\begin{description}
	\item[RQ1] 
	Conduct a thorough review of academic literature, industry reports, and government publications to collect current data on \gls{ghg} emissions from data centers globally. Queries used for this were "GHG emissions from data centers", "CO2 emissions data centers", "CO2 emissions cloud".
	
	\item[RQ2] 
	Gather data on energy consumption, cooling system efficiency, and hardware lifecycle emissions from data center operators, equipment manufacturers, and academic literature. Queries used were "server energy consumption", \enquote{data center thermal management}, \enquote{data center cooling}, \enquote{data center e-waste}, \enquote{liquid cooling}, \enquote{free cooling}, \enquote{data center PUE}
	
	\item[RQ3] 
	Gather data on current carbon footprint reduction measures employed by data center operators, such as renewable energy methods, energy efficiency initiatives, and circular economy practices. Queries used were "data center carbon footprint", "renewable energy resources", \enquote{PUE in data centers}, \enquote{circular economy}. For this question, review papers were efficient to find measures taken over the years.
	
	\item[RQ4] 
	Develop scenarios to understand how factors like cloud provider efficiency, virtualization, and resource utilization can affect the environmental impact of cloud computing compared to on-premises data centers. Bring real life scenarios of how clouds changed the industry.
	
	\item[RQ5] 
	Conduct a review of the existing literature on DCIM 
\end{description}

This thesis includes the latest publicly available research and developments on the specific topic, as some older papers were no longer relevant. The papers and publications used in this thesis are primarily related to electrical engineering, thermal engineering, and climate change. As additional sources, the thesis references blogs from major cloud providers such as AWS, Google, and Microsoft. The figure \ref{publications} shows a report from the publications used in this thesis.

\begin{figure}
	\centering
	\includegraphics[width=\textwidth]{num_of_publications_per_year.png}
	\caption{Number of publications per year used in this research}
	\label{publications}
\end{figure}

%1.	Gathering data about carbon footprint of the data centers
%2.	What causes the emissions in data centers
%2.1.	How can each of these factors be optimized to reduce emissions
%3.	Measures taken to reduce the environmental impact by outsourcing to cloud computing
%4.	Study of the software that measure the footprint and how they computate
%5.	Experiment to track the footprint of a server and make comparisons to the software computations


%Before starting conducting the research about energy and CO2 of cloud/backend processing, it became clear that the most energy consumption comes from this massive server rooms that would process data non stop, called data centers. So this study was focused on data centers, their infrastructure and the energy that these rooms were consuming. Servers are the primary powerhouses within data centers.
%In this thesis I tried to include the latest publicly available research or development of the specific topic because some old papers were not relevant anymore. There were many old publications about data centers and their energy consumption, whose outcome was already applied in real life several times or their outcome was overridden. Papers and publications used in this thesis are related to electrical engineering, thermal engineering and climate change. As a resource for this thesis I also checked the activity of main cloud providers such as AWS, Google and Microsoft. Even though these companies are working independently from each other, they have all made huge advancements in their common goal to reach net zero carbon emissions.

%Energy and CO2 are two different things, but in data center context they are closely related. Data centers do need energy to run and to keep the server rooms temperature optimal. But to generate this energy, usually electricity, tons of CO2 is released in the atmosphere. Carbon release is the outcome that should be reduced and it is causing the harmful greenhouse effect to our planet.

%To  begin with, what is the carbon footprint of a software or IT equipment. When viewed in isolation, IT looks very sustainable. But running and maintaining servers, data centers, clouds and sofware produces indirectly tons of carbon gases. Firstly, all these require electricity to run and to guarantee uptime they need UPS. In most of the locations where majority of data centers exist, electricity is generated from burning fuel or coal, thus emitting tons of CO2 in the atmosphere. RES are seen as a better approach to reducing carbon emission, but RES do not work from anywhere. 

%Another indirect relation is thermal management. Servers have stay in optimal temperature to have maximum efficiency. Maintaining this temperature all the time requires a lot of energy, and usually this comes from electricity. Many companies are using water cooling or free cooling as to reduce energy, but this is only applied in only some data centers. 

%Hardware have improved a lot. According to Moore's Law, every 18 months, technology doubles. Old data centers have already replaced all their IT equipment, especially processing equipment, with new and powerful ones. These new substitutions are more efficient in every aspect, they are faster, consume less power and take less space. But on the other hand, this replacement causes a lot of mostly metallic waste. While it looked that it is better for the environment, there are still some remains to be processed.

%So carbon footprint of processing servers is caused from electricity, thermal management and it equipment disposal. Therefore for this thesis the main queries that I looked for were "data center energy consumption", "data center cooling", "IT equipment replacement", "IT equipment recycling", "clouds energy usage". 


%Sometimes two terms, energy and power, are used interchangeably. Power is the energy consumption for unit time. They are closely related but they push different parts of the design. Heat generation depends of power consumption. Battery life, on the other hand, depends on energy consumption. But generally, power is used as shorthand for energy and power consumption, distinguishing between them only when necessary.

%Papers and publications used in this thesis are related to electrical engineering, thermal engineering and climate change.

%what is a computation?

%%%%%%%%%%%  chapter4 %%%%%%%%%%%%%%%%%%%%

\chapter{Energy used from Data Centers}\label{chap4}

The US data centers handled about 300 million Terabyte of data that consumed around 8.3 billion kWh per year in 2016, hence 27.7 kWh per Terabyte with a carbon footprint of approximately 35 kg CO2 per Terabyte of data\cite{corbett2018sustainable}.

According to a report published by the International Energy Agency\cite{ieaDataCentres} in 2023, data centers worldwide were estimated to consume approximately 1-1.5\% of the world's electricity. While this figure may seem relatively modest, it assumes significance when considering the overall global energy consumption. Additionally, as our reliance on digital services continues to grow and more devices become interconnected, the demand placed on data centers and their energy consumption is expected to increase\cite{schomaker2015energy}. Compounding this issue is the fact that a significant portion of the energy used by data centers still comes from non-renewable sources\cite{ritchie2024energy}.

In areas where coal continues to be the primary source of electricity, the environmental impact of data center operations is particularly significant\cite{finkelman2021future}. Data centers powered by coal emit a disproportionately higher amount of carbon emissions compared to those running on cleaner energy sources. As a result, even though data centers may only account for a small percentage of global energy consumption, their contribution to greenhouse gas emissions raises valid concerns.


%\begin{figure}
%	\centering
%	\includegraphics[width=\textwidth]{recalibrating_global_data_center_energy-use_estimates.jpeg}
%	\caption{Recalibrating global data center energy-use estimates\cite{doi:10.1126/science.aba3758}}
%\end{figure}

The power consumption of a data center is influenced by a variety of factors, including the hardware specifications and internal infrastructure, the computational workloads, the types of applications running, and the cooling requirements\cite{dayarathna2015data}. These variables make the overall power usage difficult to measure precisely. Additionally, the power consumption of the IT equipment, cooling systems, and power conditioning infrastructure within the data center are all closely interrelated.


\section{Energy used for running Servers}

Servers are the primary powerhouses within data centers. They process vast amounts of data, ensuring  that  our  emails,  cloud  applications, online games, and many other digital services function seamlessly\cite{techtargetMajorServer}. Their constant  activity means they're persistently consuming electricity\cite{energyinnovationMuchEnergy}, and this continuous energy draw is a direct contributor to carbon emissions, especially if the electricity source is fossil fuel-based.

The total energy consumption of a server is the collective sum of the energy consumed by all its individual hardware components. This includes the energy used by the CPU(s), memory modules, storage drives, network interfaces, power supply, cooling fans, and any other active components within the server\cite{ahmed2021review}. 


A depiction of electricity consumption is depicted in the figure \ref{component_wise_consumption} where it is clear that CPU consumes the most of energy with 32\%.

\begin{figure}
	\centering
	\includegraphics[width=\textwidth]{Component-wise-energy-consumption-of-a-server.png}
	\caption{Component-wise energy consumption of a server; Source\cite{ahmed2021review}}
	\label{component_wise_consumption}
\end{figure}

The energy consumption of each individual component is influenced by factors such as the component's power rating, utilization level, and operating conditions. The total server energy consumption can be calculated by measuring or estimating the power draw of each component and then adding them together. According to \cite{chatzipapas2015challenge} total energy estimation of the system is calculated as below (E\textsubscript{base} accounts for the un-addressable energy losses including the idle energy consumption of the server):

\begin{center}
	E\textsubscript{total} = E\textsubscript{base} + E\textsubscript{cpu} + E\textsubscript{disk} + E\textsubscript{net} + E\textsubscript{mem}
\end{center}

Monitoring the energy usage of individual server components can provide valuable insights into optimizing the server's overall energy efficiency. This can involve techniques like right-sizing hardware, implementing power management features, and optimizing workload distribution to reduce unnecessary energy consumption\cite{LIN201847}. An example would be the comparison between two different processors like ATOM and Xeon\cite{iccusaIntelAtom}. The figure \ref{servers_comparison} shows power consumption comparison of two servers.
In an Atom-based server, the memory subsystem consumes the largest portion of the overall power. The Atom processor itself has a relatively modest power draw, but the memory modules and associated components have to work harder to compensate for the processor's limited memory capabilities, resulting in high memory power consumption\cite{turley2014white}. On the other hand, in a Xeon-based server, the CPUs are the main power consumers. Xeon processors are designed for high-performance server workloads and have significantly higher power requirements compared to Atom. The CPU cores, cache hierarchy, and advanced features all contribute to the Xeon's greater power draw. While the memory subsystem of Xeon also consumes power (19\%), it is not as dominant factor as in the Atom-based system (48\%)\cite{servethehomeMemoryBandwidth}.


\begin{figure}
	\centering
	\includegraphics[width=\textwidth]{power_breakdown_across_the_components_of_two_servers.jpg}
	\caption{Power breakdown across the components of two servers\cite{dayarathna2015data}}
	\label{servers_comparison}
\end{figure}



\subsection{Server States}
In data centers, the servers are not always active, as they can be idle. Furthermore, there are two cases to consider at energy consumption of servers:

\begin{description}
	\item[Baseline power] P\textsubscript{base}, which is the power consumption when the machine is idle, includes the power consumption of the fans, CPU, memory, I/O and other motherboard components in their idle state and is often considered as a fixed value. Idle servers are troublesome because they are not efficient and they are expensive\cite{raritanIdleServer}. In 2015, it was estimated that servers were wasting 30\% of energy in idle state, but the situation has improved a lot since then\cite{computerweeklyWasteEnergy}\cite{basmadjian2012modelling}. Methods like PowerNap have been proposed\cite{powernap_server} that eliminate idle power in servers by quickly transitioning in and out of an ultra-low power state. This method reduced in average power up to 70\% for Web 2.0 servers.
	
	\item[Active power] P\textsubscript{active}, which is the power consumption due to the workload, depends on the workload of the machine and the way it utilizes CPU, memory and I/O components. Hence, the power model can be expressed as the sum of baseline power and active power\cite{JIN2020114806}. 
	
	\begin{center}
		P\textsubscript{total} = P\textsubscript{base} + P\textsubscript{active}
	\end{center}
\end{description}


\subsection{Backup Generators}
Backup generators play a crucial role in ensuring the operational reliability of data centers. These generators, powered by diesel fuel, are responsible for generating electricity during power outages or routine testing. However, it is important to note that when these generators are activated, they produce direct emissions that can have an impact on the environment. While their operational hours may be limited compared to primary equipment, the combustion of fuel leads to a higher release of \gls{ghg}\cite{jiang2015energy}.

\section{Energy used for Thermal Management}\label{thermal_energy_section}

Historically, data centers have heavily relied on traditional \gls{hvac} systems to maintain desired temperatures. However, these systems are known for their high energy consumption. The main reason for this is their use of mechanical cooling, which involves compressors and refrigerants that consume significant amounts of energy, sometimes even matching the energy consumption of the servers themselves\cite{bhatia2015hvac}.

So, in addition to the energy required to power the servers, a substantial amount of energy is dedicated to cooling them and maintaining optimal operating temperatures\cite{zhang2021survey}. This is necessary because the processors within the servers generate a significant amount of heat during operation. Consequently, server rooms tend to become hot due to the heat produced. To address this challenge, numerous measures have been implemented to effectively manage heat distribution and ensure efficient cooling within data centers\cite{zhang2021survey}. Figure \ref{servers_cooling_system} is a design configuration of underfloor air distribution system uses a raised floor plenum to supply conditioned air into a space. The air conditioning equipment, such as \gls{crac} units, is located around the room's perimeter. The conditioned air is delivered upwards through vents or diffusers in the raised floor.


\begin{figure}
	\centering
	\includegraphics[width=\textwidth]{Cooling_System_with_Raised_Floor_Configuration.png}
	\caption{Cooling System with Raised Floor Configuration (hot aisle/cold aisle layout)\cite{bhatia2015hvac}}
	\label{servers_cooling_system}
\end{figure}

Operating computing systems for extended periods of time at high temperatures
greatly reduces reliability, longevity of components and will likely decrease QoS. \gls{ashrae} has in fact now recommended an acceptable operating temperature range of 18° to 27°C (64° to 81°F) to be optimal for system reliability. For environments with low levels of both copper and silver corrosion, the recommended temperature range is between 18° and 21°C (64 to 69.8 °F). However, server manufacturer Dell states that the temperature “sweet spot” for their servers is 26.7°C (80°F)\cite{AVTECH}. Expensive \gls{it} equipment should not be operated in computer rooms or data centers where the ambient room temperature exceeds  30°C (85°F)\cite{42U_2016}.

It is worth noting that the excessive energy consumption associated with traditional \gls{hvac} systems has prompted the exploration and adoption of more sustainable and energy-efficient cooling solutions in modern data centers\cite{nadjahi2018review}. These innovative approaches aim to reduce both the environmental impact and operational costs of data center cooling. By employing advanced cooling techniques, such as liquid cooling, airflow optimization, and heat recycling, data centers can achieve greater energy efficiency and minimize their carbon footprint.

According to \cite{ZHU2023104322} approximately 15–27\% of the energy consumed by \gls{dc} can be reduced via research and development of advanced cooling technologies, including natural cooling and liquid cooling to achieve sustainability.

Below are some strategies and layout designs that improve air circulation in the server rooms:
\begin{description}
	
	\item[Enhancing Natural Ventilation and Thermal Efficiency]
	The need for artificial cooling can be reduced by employing designs that prioritize natural ventilation, use thermally conductive materials, and optimize server layout. Consequently this would enable efficient airflow and would improve temperature regulation in data centers. On the other hand, materials with high thermal conductivity also help dissipate heat more effectively. The combination of natural ventilation and thermally conductive materials creates an energy-efficient \gls{dc} environment\cite{zhang2021optimization}\cite{stavreva2023best}
	
	%%% elaborate this more

	\item[Utilizing Energy-Efficient Architectural Features]
	Implementing certain architectural designs can further enhance the energy efficiency of a data center\cite{moazamigoodarzi2019influence}. For instance, roofs painted with reflective coatings help to reduce heat absorption from the sun, minimizing the need for additional cooling\cite{akbari2003measured}. Green roofs, which incorporate vegetation, provide natural insulation and cooling effects, reducing the overall energy demand\cite{susca2011positive}. Additionally, incorporating thermal buffers like double-wall constructions can act as a barrier against external temperature fluctuations, helping to maintain a more stable internal environment\cite{ewim2023impact}.

	\item[Innovative Layout Strategies]
	Hot/cold aisle containment is a cooling strategy for data centers. It involves arranging server racks so that the hot air exhausts are directed one way, and the cold air intakes are directed the other way. This separation of hot and cold airflows helps prevent hot air from being recirculated back into the cold aisle\cite{jin2020case}\cite{jin2019effects}.
	
	Another layout strategy being explored is vertical server stacking. This configuration takes advantage of the natural physics of heat rising, as hot air naturally moves upward. By stacking servers vertically and strategically placing vents, the vertical arrangement can facilitate better heat dissipation and improve cooling efficiency\cite{jin2020case}\cite{jin2019effects}. Figure \ref{server_layout} illustrates the stacking of servers and airflow. A carefully implemented hot aisle/cold aisle layout can dissipate heat loads of up to around 5 kW per rack, but older data centers without optimized cooling typically only manage 1-2 kW per rack\cite{bhatia2015hvac}.
	
	\begin{figure}
		\centering
		\includegraphics[width=\textwidth]{Hot_aisle_Cold_aisle_Layout.png}
		\caption{Hot aisle- Cold aisle Layout )\cite{bhatia2015hvac}}
		\label{server_layout}
	\end{figure}
	
\end{description}

By implementing these design principles and layout strategies, data centers can significantly reduce their energy consumption, enhance cooling efficiency, and contribute to a more sustainable and environmentally friendly operation.

The strategies for achieving \gls{ecer} in different districts \gls{dc}s can vary based on factors like geographical location, natural resources, and economic foundations. A single energy-saving approach may not be sufficient to meet the goal of zero-carbon emission across all \gls{dc}s\cite{malkamaki2012solar}.


\subsection{Free Cooling}
There are several types of temperature management methods employed in servers and data centers, with air cooling being the most commonly utilized. Air cooling involves the use of fans that generate airflow to cool down the servers and distribute cold air to the hot components. Another cooling technology is free cooling\cite{data4groupCloserLook}, which represents a more recent development in this field.

Data centers in very cold regions can use the ambient outdoor temperature to help maintain a consistent temperature inside the server rooms\cite{data4groupCloserLook}. By harnessing the naturally low temperatures of their surroundings, these data centers are able to reduce their reliance on traditional cooling mechanisms. Data centers have several options for free cooling approaches. One is airside free cooling, which uses outside air to directly cool the data center without mechanical refrigeration. Another is waterside free cooling, which relies on a water-based heat exchange system to leverage natural cold water sources. The third approach is heat pipe free cooling, which utilizes a network of heat pipes to transfer heat from the data center to the outside. 

The choice of free cooling method depends on factors like the local climate, water availability, and the facility's specific cooling requirements. Data centers that are situated in warmer climates or regions lacking a sufficiently cold environment may not be able to fully utilize this method. In such cases, many data centers end up using a combination of these free cooling techniques to enhance energy efficiency and to ensure optimal temperature management and prevent overheating of the servers and related equipment\cite{zhang2014free}.

Free cooling approach not only results in lower energy consumption compared to conventional air cooling, but it also minimizes the environmental impact associated with cooling operations\cite{datacenterdynamicsExploringData}.

Examples of data centers that use free cooling are:
\begin{itemize}
	\item Google has built its data centers in Hanima, where they use seawater from the Bay of Finland for the cooling system\cite{datacentersGoogleData}
	\item Microsoft has drowned 864 servers 22 km from the Scottish coast and uses local cooling to reduce energy consumption\cite{qzNewestGreentech}
\end{itemize}



\subsection{Liquid Cooling}

Liquid cooling systems use fluids to dissipate heat more effectively than air cooling. The cooling capacity of liquid systems is typically 1000-3000 times greater than air cooling. This improved heat transfer can enable slightly lower energy consumption in some applications that utilize liquid cooling.\cite{ZHU2023104322}.


%\begin{figure}
%	\centering
%	\includegraphics[width=\textwidth]{Single-phase_submerged_liquid_coolingTwo-phase_submerged_liquid_cooling.jpg}
%	\caption{Single-phase submerged liquid cooling, Two-phase submerged liquid cooling.; Source: \cite{li2023china}}
%\end{figure}

Liquid cooling is an advanced temperature management technique that surpasses the efficiency of air cooling. While the majority of current IT equipment is designed for air cooling, liquid cooling offers several notable advantages\cite{xu2023thermal}.

\begin{description}
	\item[Indirect liquid cooling] This is a method of heat removal where the heat source and liquid coolant do not directly interact. In this approach, the traditional air-cooled heat sink must be replaced with a liquid-cooled alternative, such as an evaporator. This change allows for more efficient heat transfer from the source to the coolant, and can result in smaller component sizes and greater heat dissipation capabilities compared to air cooling\cite{khalaj2017review}. Figure \ref{indirect_cooling} shows a schematic diagram of this kind of cooling technology.
	
	\begin{figure}
		\centering
		\includegraphics[width=\textwidth]{indirect_water_cooling_system.png}
		\caption{The schematic diagram of indirect water cooling system; Source: \cite{khalaj2017review}}
		\label{indirect_cooling}
	\end{figure}
	
	\item[Direct liquid cooling] In this approach, the coolant flows directly over the surfaces of electronic components. This allows for highly efficient heat transfer by eliminating the additional thermal interface present in indirect liquid cooling systems. The coolants used are typically dielectric fluids - electrically insulating liquids that can safely make direct contact with active electronics, providing both effective cooling and electrical isolation\cite{khalaj2017review}\cite{submerImmersionCooling}. One example, immersion cooling is 1400 times better thermal conductor than air\cite{submerImmersionCooling} which translates to being way more energy efficient. As technological advancements continue, it is expected that liquid cooling will become more prevalent and feasible for a wider range of applications.
	
\end{description}

It's important to acknowledge that maintaining a liquid cooling system presents certain challenges\cite{azarifar2024liquid}. In the case of dielectric cooling, the management of the liquid becomes more complex\cite{submerImmersionCooling}. To implement dielectric cooling effectively, substantial containers capable of accommodating all the components must be created. Such large-scale implementation is currently limited, with testing primarily conducted on smaller servers.


%%%%%%%%%%%%%   Chapter5   %%%%%%%%%%%%%%%

\chapter{Hardware Footprint and E-waste}\label{chap5}
%The IT equipment consume approximately 50\% of the DC energy \(Rteil, Bashroush, \& Kenny, 2022).
It is estimated that manufacturing storage devices has resulted in 20 million metric tonnes of CO2 emissions in 2021 alone\cite{networkworldHDDsGreener}.

Data centers are comprised of a vast amount of hardware which are categorized into two distinct types of equipment: data center infrastructure and computing components. The data center infrastructure consists of a wide range of mechanical and electrical components, such as transformers, generators, air conditioners, racks, and cables. These components work together to support the functioning of the data center. On the other hand, the computing components refer to the servers that handle the processing, storage, and retrieval of data. To ensure optimal performance, the equipment in this category is periodically replaced. Typically, infrastructure components are replaced every 10 to 15 years, taking into account advances in technology, energy efficiency, and the evolving demands of the data center environment\cite{samaye2024energy}. In contrast, computing components are more frequently changed. This process generates a significant amount of metallic waste, contributing to a substantial environmental footprint. Fortunately, advancements in technology have led to the development of smaller and more efficient hardware solutions.  

One notable advancement is the improvement in server design\cite{hoorn2016improving}. Modern servers not only offer better performance but also prioritize energy efficiency. They are designed to provide more computational power per watt, reducing overall energy consumption. Additionally, the adoption of \gls{ssd} instead of traditional hard drives and the utilization of energy-efficient processors have further optimized energy usage within data centers. A 1 terabyte (TB) \gls{hdd} is estimated to consume around 184 kWh of electrical energy over a 5-year period. In comparison, a 1TB \gls{ssd} would only consume approximately 57 kWh over the same timeframe\cite{networkworldHDDsGreener}. 

Implementing intelligent operations and maintenance management for the cooling system, \gls{it} equipment, and lighting in a data center in Langfang resulted in a 1.2\% reduction in the \gls{pue} value. This implementation also led to an annual electricity saving of 33.36 million kWh\cite{li2023china}.

%\begin{figure}
%	\centering
%	\includegraphics[width=\textwidth]{pue_range2012-2019.jpg}
%	\caption{The range of PUE; Source: IDC Foresight Industry Research Institute}
%\end{figure}

Furthermore, innovative techniques such as virtualization have played a crucial role in reducing redundant hardware operations\cite{hivenetEWasteCloud}. Virtualization allows a single physical server to operate as multiple virtual servers, thereby reducing the need for numerous physical servers and minimizing energy consumption. Additionally, technologies like data deduplication, which identifies and stores redundant data only once, have optimized storage capacity and further reduced energy requirements.

While recycling is necessary to manage the waste generated by data centers, it is important to acknowledge that it also has its own environmental impact, illustration figure \ref{recycling_lifecycle}. The transportation of waste to recycling facilities contributes to energy consumption and emissions\cite{nor2019energy}. To address this, data center operators are exploring ways to improve recycling processes, such as implementing local recycling facilities or utilizing more energy-efficient transportation methods.

\begin{figure}
	\centering
	\includegraphics[width=\textwidth]{dc_component_recycling_process.png}
	\caption{IT Equipment Recycling Process; Source: Sims Lifecycle Services\cite{simslifecycleEquipmentProcessed}}
	\label{recycling_lifecycle}
\end{figure}


Furthermore, the concept of circular economy is gaining traction in the data center industry. This approach emphasizes the reuse and repurposing of hardware components whenever possible, reducing the need for constant replacement and minimizing waste generation. By adopting circular economy principles, data centers can significantly reduce their environmental footprint\cite{datacentremagazineNavigatingAddressing}.

Technological advancements in server design, the reduction of redundant hardware operations through virtualization, and the adoption of sustainable practices such as recycling and the circular economy are key factors in mitigating the environmental impact of hardware components. By continuously striving for energy efficiency, waste reduction, and responsible resource management, the data center industry can contribute to a more sustainable and eco-friendly future.


%%%%%%%%%%%%%   Chapter6   %%%%%%%%%%%%%%%%%%%

\chapter{Clouds and Virtualization}\label{chap6}

In the early days, industries relied on physical, role-based servers. These servers posed several challenges. Firstly, scaling them according to the workload they received was difficult. As the demand on the servers increased, it was hard to expand the infrastructure to accommodate the growing load. Secondly, managing the infrastructure of these servers was a complex task. The maintenance, configuration, and overall upkeep required significant time and resources. Additionally, if one of these physical servers failed, it would directly impact the corresponding service, resulting in downtime and disrupted operations. Due to these challenges, companies started considering alternative approaches to tackle them, which eventually gave rise to the concept of outsourcing the infrastructure and maintenance of servers. This led to the development of cloud computing, where organizations could ship these responsibilities and concentrate on their core operations, benefiting from remote server management. As of today, Cloud computing is the fastest growing internet technology, \ref{cloud_increase}\cite{anand2021need}. Berkley RAD lab defines "Cloud Computing refers to both the applications delivered as services over the Internet and the hardware and systems software in the datacenters that provide those services. The services themselves have long been referred to as Software as a Service (SaaS), so we use that term. The \gls{dc} hardware and software is what we will call a Cloud"\cite{fox2009above}.
Clouds perform better in terms of CPU utilization, latency, scalability which in the long run translates to less energy use \cite{khanghahi2013cloud}. 


\begin{figure}
	\centering
	\includegraphics[width=\textwidth]{Public_cloud_services_end_user_spending_worldwide_from_2017_to_2024.png}
	\caption{Public cloud services end-user spending worldwide from 2017 to 2024\cite{cloud_economy2023}}
	\label{cloud_increase}
\end{figure}


Virtualization is a key technology in cloud computing that brings numerous advantages. It offers increased flexibility, dynamic resource allocation, and improved resource utilization. By consolidating applications onto fewer physical servers, virtualization helps reduce power consumption and the need for cooling systems. Data center operators can optimize energy usage and pricing of cloud computing platforms by employing suitable \gls{vm} power models. Results from \cite{6679892} show that a data center using the proposed task-scheduling scheme consumes on average over 70 times less on server energy than a data center using a random-based task-scheduling scheme. Clouds use various approaches and strategies to optimize cloud performance\cite{alzakholi2020comparison}.

The adoption of cloud computing has resulted in a significant annual reduction of 20\% in the energy intensity of data centers since 2010. The benefits of energy-optimized and cost-efficient cloud data centers are now accessible to individuals globally through cloud computing services like Alibaba Cloud, \gls{aws}, Google Cloud, and Microsoft Azure. The widespread availability of cloud computing platforms allows people from diverse backgrounds to conveniently access computing resources. This enables them to benefit from efficient and sustainable cloud infrastructure.\cite{patterson2021carbon}.

There have been meaningful discussions regarding the impact of cloud computing on the low-carbon economy\cite{tao2022can}. Cloud computing technology has been recognized for its contribution to the evolution of the industrial structure towards low energy consumption and the facilitation of a low-carbon economy. By consolidating computing resources and implementing energy-efficient technologies, cloud computing significantly reduces energy consumption compared to traditional on-premises infrastructure\cite{shahbaz2022impact}. On the other side, there are those who believe that the said technological innovation has led to the creation of large-scale data centers and extensive information infrastructures.

Cloud evaluation is difficult compared to systems as many factors have to be taken into consideration like configurations, connectivity and liveliness. But in terms of cloud data centers as a whole, big companies like Google, Amazon, Microsoft have done many research to be efficient  more than any other local system provider\cite{google-datacenters}. Large data centers try to minimize their \gls{pue} ratio, with the current average being 1.5. Many new data centers from major cloud providers and colocation facilities are more efficient than older facilities. Google is considered one of the most power-efficient cloud providers, with a PUE ratio of 1.1 as shown in figure \ref{clouds_pue_img}\cite{clouds_pue}.

\begin{figure}
	\centering
	\includegraphics[width=\textwidth]{statistic_id1362483_prominent-cloud-provider-annual-power-usage-effectiveness--pue--worldwide-2022.png}
	\caption{Prominent data center operators and cloud providers annual power usage effectiveness (PUE) in 2022\cite{clouds_pue}}
	\label{clouds_pue_img}
\end{figure}


%% add graph of PUE for big cloud providers

%%%%%%%%%%%%%%   Chapter7   %%%%%%%%%%%%%%%

\chapter{Energy saved from Data Centers/Clouds}\label{chap7}

Reducing the carbon footprint on your project should not increase the carbon footprint elsewhere. This means that outsourcing to data centers and clouds does not necessarily reduce carbon footprint. The impact can vary depending on factors like energy sources and workloads.

The increasing adoption of \gls{ict} across various sectors of the economy has resulted in a significant rise in energy consumption from data centers and clouds. This upward trend can be characterized as exponential, reflecting the growing reliance on \gls{ict} for enhanced productivity and efficiency.

Research focusing on emerging economies has uncovered a noteworthy relationship between Internet usage and electricity consumption. Specifically, a 1\% increase in Internet users has been found to correspond to an increase per capita electricity consumption of up to 0.36\%\cite{rahimi2017internet}. This indicates that as more individuals gain access to the internet and engage in online activities, the demand for energy to power the necessary \gls{ict} infrastructure also escalates.

Conversely, studies conducted in \gls{oecd} countries have revealed an interesting finding. They suggest that a 1\% increase in \gls{ict} capital can lead to a reduction in energy demand by 0.235\%\cite{lange2020digitalization}. However, it is important to note that this reduction is primarily attributed to a decrease in the consumption of non-electric energy sources rather than electricity consumption itself. This implies that the implementation of \gls{ict} technologies in these sectors has resulted in energy efficiency gains in non-electric energy usage.


In a study, researchers discovered energy implications of various technological shifts. The study found that transitioning from conventional newspapers to online newspapers had the potential to yield energy savings of up to 60\%\cite{lange2020digitalization}. These findings align with previous research that demonstrated similar energy savings when comparing music downloads to physical CD delivery. Additionally, the adoption of e-readers was identified as another avenue for achieving substantial energy savings.

The study also uncovered contrasting results in different areas. For instance, it was discovered that Internet game downloads resulted in higher \gls{ghg} emissions compared to physical Blu-ray discs\cite{lange2020digitalization}. Similarly, when examining the energy savings in e-commerce, the findings were mixed. A study focusing on the Japanese book sector revealed that e-commerce was as energy-intensive as conventional retailing in rural areas, but more energy-intensive in urban areas.

The researchers identified several key factors that influence energy consumption in e-commerce, including population density, freight mode, product return rate, trip allocation, and packaging type\cite{zaharia2019factors}. They also compared the energy impact of paper book sales in traditional and online bookshops \cite{lange2020digitalization} and concluded that online bookshops were marginally more beneficial. The study emphasized that the perspective of the buyer, including their means of transport and combination with other activities, plays a significant role in determining the overall outcome.

Interestingly, despite the availability of video conferencing systems\cite{lange2020digitalization}, the number of international scientific conferences has been on the rise. Similarly, the demand for printed books has remained steady, while e-books and online reading materials have experienced increased popularity. Although video streaming can offer energy savings compared to physical DVD purchases or rentals, the growing hours of streaming and data traffic associated with video streaming are expected to offset these potential savings.

Shifting focus to processors, the study found that the energy intensities of processing units (CPUs) decrease by approximately half every 1.5 years\cite{lange2020digitalization}. However, Moore's Law\cite{schaller1997moore} suggests that processing capacities also double within the same timeframe. Consequently, the potential energy savings resulting from increases in energy efficiency are counterbalanced by the rising demand for processing services. The growth in scientific conferences, books, data traffic, and processing capacities has been observed, but the reasons behind this growth are not entirely clear. It is uncertain to what extent increased energy efficiency has contributed to these increases, as other factors may also be at play. More investigation may be needed to better understand the relative influence of improved energy efficiency versus other possible drivers of the observed growth in these scientific and technological metrics.

Additionally, approximately 97\% of the energy consumed by \gls{dc}s can be converted into heat and reused\cite{distributed_head}.

\section{Carbon-Free Energy Data Centers}

To have carbon free energy, electricity should be produced from renewable resources\cite{qazi2019towards}. The availability of solar and wind energy is influenced by both investment and weather conditions. But wind doesn't always blow, sun doesn't always shine. Thats when utility-scale batteries come to power that can actually support the grid with continuous energy.

According to an official statement from Google, they have announced their commitment to achieving carbon-free energy across all their data centers and campuses worldwide. This ambitious goal ensures that their facilities consistently rely on clean energy sources, without any carbon emissions, to power the essential products and services that users depend on daily. As a result, whether you're sending an email via Gmail, making inquiries through Google Search, watching YouTube videos, or navigating with Google Maps, rest assured that each of these activities will be supported by renewable energy around the clock\cite{google2020thirddecade}.

%%%%%%%%%%%%%%%%%%  Chapter8  %%%%%%%%%%%%%%%%%%%

\chapter{DCIM Software for Carbon Footprint Measuring}\label{chap8}
\gls{dcim} systems connect IT and facility management in data centers. They create a shared database of IT and facility assets, linking the IT side to the facility side\cite{salimenergy}. \gls{dcim} tools allow real-time monitoring and optimization of data center cooling systems, helping to maintain appropriate temperatures for IT equipment. Meanwhile, data centers are enhancing their electricity infrastructure through measures like improved power distribution, backup power, and renewable energy integration\cite{murino2023sustainable}. One of the key capabilities is measuring and tracking the carbon footprint of the data center. DCIM software collects, trends, and reports on data so that data center managers can see their carbon footprint by location, set a target, and track their performance over time\cite{sunbirddcimDataCenter}. \ref{dcim_dashboard} is example of \gls{dcim} dashboard.


\begin{figure}
	\centering
	\includegraphics[width=\textwidth]{Overview_of_dcim_dashboard.png}
	\caption{An example: Environmental monitoring dashboard in netTerrain DCIM\cite{graphicalnetworksLoweringYour}}
	\label{dcim_dashboard}
\end{figure}

%%%%%%%%%%%%%%%%%%  Chapter 9  %%%%%%%%%%%%%%%%%%%

\chapter{Results}

\begin{description}
	\item[RQ1]
	
	This question is answered throughout the whole thesis. In each section we have something related to how much \gls{ghg} is produced. And we also have an estimation of 720 million tones of CO2 emission in 2030.
	\gls{ghg} is an essential term in this whole research. In 2022 data centers produced 3.7\% of all \gls{ghg} worldwide. It is estimated that by 2030 \gls{dc} will produce 720 million tones of CO2.
	
	\item[RQ2]
	
	This question is studied in chapters \ref{chap4} and \ref{chap5}
	
	\item[RQ3] 
	
	Carbon footprint is introduced in section \ref{ghg_section_background}. The measures to reduce carbon footprint are elaborated in different chapters. In section \ref{thermal_energy_section} there are given strategies to improve thermal management in \gls{dc}. While in chapter \ref{chap5} it is studied the concept of circular economy and recycling.
	
	\item[RQ4] 
	
	The efficiency of cloud providers in utilizing resources is discussed in Chapter \ref{chap6}. Chapter \ref{chap7} examines the energy savings and reduced CO2 emissions that result from outsourcing to cloud data centers.
	
	\item[RQ5] 
	
	\gls{dcim} software is studied on chapter \ref{chap8}. There was no experiment done to evaluate \gls{dcim}.
	
\end{description}


%\blinddocument

%\chapter{Related Work}

%Describe relevant scientific literature related to your work.

\chapter{Conclusion}
\label{chap:zusfas}
Over the past year, the demand for data storage and processing has been steadily increasing. Correspondingly, the demand for data centers has also risen significantly. Currently, data centers consume around 3.7\% of the world's total energy, surpassing even the energy usage of the aircraft industry.

However, despite this growing number of data centers, there has also been a notable improvement in data processing efficiency. Today, data centers are able to achieve the same level of data processing while consuming less energy compared to the previous year. A 6\% increase in data center energy in 2018, translated to 550\% increase in compute instances. Expressed as energy use per compute instance, the energy intensity of global data centers has decreased by 20\% annually since 2010\cite{doi:10.1126/science.aba3758}. This increased efficiency has helped offset some of the impacts of the rise in the number of data centers.

 As our reliance on digital technologies continues to grow, finding ways to make data centers more sustainable will be crucial going forward. 
 
The following steps are essential:
\begin{itemize}
	\item Accelerate research and development of cloud computing technologies and improve the power usage effectiveness (PUE) of cloud infrastructure.
	\item Enhance the management of computing, storage, and networking resources within data centers to optimize energy usage.
	\item Invest in and integrate renewable energy resources, such as solar and wind power, to power data center operations in a sustainable manner.
	\item Implement advanced thermal management techniques to improve cooling efficiency in data centers.
	\item Focus on the overall architectural design of data centers to integrate energy-efficient principles from the ground up.
	\item Ensure hardware components are recycled in an environmentally responsible manner.
	\item Raise public awareness about the energy consumption of the internet and encourage more mindful internet usage habits.
\end{itemize}


\printbibliography

All links were last followed on May 21, 2024.

%\appendix
%\input{latexhints-english}

\pagestyle{empty}
\renewcommand*{\chapterpagestyle}{empty}
\Versicherung
\end{document}
